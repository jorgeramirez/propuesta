\section{Introducci\'{o}n}
\label{sec:intro}

El habla, y m\'{a}s concretamente el lenguaje como medio de comunicaci\'{o}n, es una de las 
caracter\'{i}sticas fundamentales 
que diferencian al ser humano de los dem\'{a}s animales y representa un factor clave de su evoluci\'{o}n. 

El reconocimiento del habla (tambi\'{e}n conocido como reconocimiento autom\'{a}tico del habla) 
es el proceso de convertir
una se\~{n}al de habla en una secuencia de palabras, mediante un algoritmo implementado como 
programa computacional \citet{JaisalAReview2012}. Su integraci\'{o}n con interfaces de usuario busca una interacci\'{o}n 
humano-computadora m\'{a}s natural (humana), y as\'{i} poder eliminar las limitaciones existentes
en el modelo convencional de interacci\'{o}n. Su integraci\'{o}n con las 
interfaces de usuario presenta nuevas oportunidades para buscar soluciones a problemas afectados por las 
limitaciones del modo convencional de interacci\'{o}n humano-computadora.

El reconocimiento del habla despierta actualmente gran inter\'{e}s como \'{a}rea de investigaci\'{o}n, 
en buena medida debido a las numerosas aplicaciones relacionadas, entre las cuales pueden citarse:

\begin{itemize}
    \item Medicina y Derecho: dictado autom\'{a}tico utilizado para la transcripci\'{o}n de 
    diagn\'{o}sticos y recetas m\'{e}dicos y textos legales [2].

    \item Milicia: sistemas de mando basados en reconocimiento de comandos de voz para los aviones 
    de combate de alto desempeño [3] y [4].

    \item Telefon\'{i}a: automatizaci\'{o}n de los servicios de operadora, automatizaci\'{o}n de 
    asistencia de directorio telef\'{o}nico, marcaci\'{o}n por voz [5].

    \item Dom\'{o}tica: encendido de luces, acceso a determinados lugares, activaci\'{o}n de alarmas, etc. 
    mediante comandos de voz [6].

    \item Accesibilidad: sistemas orientados a personas con discapacidades [7].

    \item Industria automotriz: control de accesorios del autom\'{o}vil como tel\'{e}fono, radio, 
    reproductor de CDs, sistemas de navegaci\'{o}n, ventanillas, entre otros [8].
\end{itemize}


Las ventajas que presenta una interfaz de usuario basada en reconocimiento del habla 
frente a una convencional son:

\begin{itemize}
    \item Velocidad: la mayor parte de la gente puede pronunciar m\'{a}s de 200 palabras por minuto, 
    mientras muy pocas pueden tipear m\'{a}s de 60 palabras en el mismo tiempo.

    \item Flexibilidad: el usuario podr\'{i}a interactuar con la computadora en la oscuridad, 
    o sin estar sentado, entre otras situaciones [9].
    
    \item Naturalidad: utilizar el habla en las operaciones de comando y control suponen una 
    sensaci\'{o}n m\'{a}s humana de interacci\'{o}n para el usuario. 
    
    \item Robustez y Precisi\'{o}n: prove\'{i}das en la comunicaci\'{o}n para diferentes usuarios 
    en diferentes entornos[6].
    
    \item Reducci\'{o}n de Costos: un sistema de dictado autom\'{a}tico puede llevar a un ahorro 
    significativo en comparaci\'{o}n a un transcriptor humano [2].
    
    \item Accesibilidad: personas con cierta discapacidad podr\'{a}n interactuar con la computadora.
\end{itemize}