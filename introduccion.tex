\section{Introducci\'{o}n}
\label{sec:intro}

El habla, y m\'{a}s concretamente el lenguaje como medio de comunicaci\'{o}n, es una de las 
caracter\'{i}sticas fundamentales 
que diferencian al ser humano de los dem\'{a}s animales y representa un factor clave de su evoluci\'{o}n \cite{SchepartzLanguage1993}. 

El reconocimiento del habla, tambi\'{e}n conocido como reconocimiento autom\'{a}tico del habla,
es el proceso de convertir
una se\~{n}al de voz en una secuencia de palabras, mediante un algoritmo implementado 
programáticamente \cite{JaisalAReview2012}. Su integraci\'{o}n con interfaces de usuario busca una interacci\'{o}n 
humano-computadora m\'{a}s natural, de manera a superar las limitaciones existentes 
en el modelo convencional de interacci\'{o}n.

El reconocimiento del habla despierta actualmente gran inter\'{e}s como \'{a}rea de investigaci\'{o}n, 
en buena medida, debido a las numerosas aplicaciones relacionadas, entre las cuales pueden citarse:

\begin{itemize}
    \item Medicina y Derecho: dictado autom\'{a}tico utilizado para la transcripci\'{o}n de 
     diagn\'{o}sticos y recetas m\'{e}dicas y textos legales \cite{GrassoTheLong2003}.

    \item Milicia: sistemas de mando basados en reconocimiento de comandos de voz para aviones 
     de combate de alto desempe\~{n}o \cite{Eurofighter} y \cite{WeinsteinOpportunities1991}.

    \item Telefon\'{i}a: automatizaci\'{o}n de los servicios de operadora, automatizaci\'{o}n de 
     asistencia de directorio telef\'{o}nico o marcaci\'{o}n por voz \cite{RabinerAutomatic1997}.

    \item Dom\'{o}tica: encendido de luces, acceso a determinados lugares, activaci\'{o}n de alarmas, 
     mediante comandos de voz \cite{BellesiReconocimiento2008}.

    \item Accesibilidad: sistemas orientados a personas con discapacidad \cite{MirandaUnJuego2007}.

    \item Industria automotriz: control de accesorios del autom\'{o}vil como tel\'{e}fono, radio, 
     reproductor de CDs, sistemas de navegaci\'{o}n, ventanillas, entre otros \cite{SalaSpeech1999}.
\end{itemize}


Las ventajas que se atribuyen a una interfaz de usuario basada en reconocimiento del habla 
frente a una convencional son:

\begin{itemize}
    \item Velocidad: la mayor parte de la gente puede pronunciar m\'{a}s de 200 palabras por minuto, 
     mientras muy pocas pueden tipear m\'{a}s de 60 palabras en el mismo tiempo \cite{RufinerSistema2004}.

    \item Flexibilidad: el usuario podr\'{i}a interactuar con la computadora en la oscuridad, 
     o sin estar sentado, entre otras situaciones \cite{RufinerSistema2004}.
    
    \item Naturalidad: utilizar el habla en las operaciones de comando y control supone una 
     sensaci\'{o}n m\'{a}s humana de interacci\'{o}n para el usuario \cite{BellesiReconocimiento2008}. 
    
    \item Robustez y Precisi\'{o}n: prove\'{i}das en la comunicaci\'{o}n para diferentes usuarios 
     en diferentes entornos \cite{BellesiReconocimiento2008}.
    
    \item Reducci\'{o}n de Costos: un sistema de dictado autom\'{a}tico puede llevar a un ahorro 
    significativo en comparaci\'{o}n a un transcriptor humano \cite{GrassoTheLong2003}.
    
    \item Accesibilidad: permite a personas con cierta discapacidad interactuar con la  computadora, 
     lo cual resulta dif\'{i}cil o imposible a trav\'{e}s de una interfaz de usuario tradicional \cite{GarretUsing2011}.
\end{itemize}