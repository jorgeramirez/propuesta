\section{Metodolog\'{i}a}
\label{sec:metodologia}

La metodolog\'{i}a a utilizar en este trabajo se basa en los objetivos espec\'{i}ficos que se 
han planteado. En primer lugar se realizar\'{a} una investigaci\'{o}n sobre el aspecto te\'{o}rico 
del reconocimiento del habla y los avances que se dieron en esta \'{a}rea hasta la actualidad.

A partir de la investigaci\'{o}n realizada, se pretende llevar a cabo un an\'{a}lisis y 
caracterizaci\'{o}n del proceso gen\'{e}rico o modelo b\'{a}sico de reconocimiento del habla. 
Este modelo b\'{a}sico sirve como punto de partida para los procesos utilizados en los sistemas actuales.

El siguiente paso en este proceso de investigaci\'{o}n y desarrollo ser\'{i}a la evaluaci\'{o}n de las 
distintas herramientas que se han desarrollado para implementar soluciones con 
capacidad de reconocimiento del habla.

Posteriormente se realizar\'{a} el dise\~{n}o, implementaci\'{o}n y evaluaci\'{o}n de una interfaz mediante
voz del usuario que permita mejorar el acceso a una aplicaci\'{o}n determinada. Esto permitir\'{a}, por ejemplo, 
que personas con ciertos tipos de discapacidad utilicen el software. 

Esta etapa de implementaci\'{o}n adoptar\'{a} la metodolog\'{i}a de trabajo de C\'{o}digo
Abierto \cite{WikipediaOpenSource}, lo cual implica:  

\begin{itemize}
    \item Se utilizar\'{a}n herramientas de C\'{o}digo Abierto para el desarrollo de la interfaz. 

    \item El proceso de desarrollo ser\'{a} transparente y abierto. 
    
    \item La interfaz desarrollada, junto con su c\'{o}digo fuente, estar\'{a} disponible de manera libre 
    y gratuita para la comunidad de C\'{o}digo Abierto.
\end{itemize}

Finalmente, luego de haber evaluado la interfaz, se expondr\'{a}n los resultados experimentales y las 
conclusiones del trabajo de investigaci\'{o}n y desarrollo.
