\section{Justificaci\'{o}n}
\label{sec:just}

El habla representa el principal modo de comunicaci\'{o}n y la forma m\'{a}s eficiente y natural de 
intercambio de informaci\'{o}n entre seres humanos. Por tanto, resulta l\'{o}gico el desarrollo 
de tecnolog\'{i}as de reconocimiento del habla para la interacci\'{o}n humano-computadora \cite{GaikwadAReview2010}. 

Teniendo esto en cuenta, se propone el dise\~{n}o e implementaci\'{o}n de una interfaz mediante voz 
del usuario para la utilizaci\'{o}n de una aplicaci\'{o}n determinada, adem\'{a}s de la 
realizaci\'{o}n de pruebas experimentales que permitan evaluar su grado de precisi\'{o}n y usabilidad.

El programa mencionado presenta un medio para aplicar y exponer los conocimientos te\'{o}ricos 
adquiridos mediante la investigaci\'{o}n previa y, a la vez, explorar y evaluar el impacto y 
los beneficios de modos alternativos de interacci\'{o}n humano-computadora.

Adem\'{a}s, se considera este trabajo como una oportunidad para abrir l\'{i}neas de investigaci\'{o}n 
relacionadas al reconocimiento del habla y el procesamiento del lenguaje, as\'{i} como tambi\'{e}n a 
la interacci\'{o}n humano-computadora y la accesibilidad, pudiendo servir como base para un 
gran n\'{u}mero de trabajos futuros.
